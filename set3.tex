\documentclass{article}
\usepackage{amsmath,amsthm,amssymb}

\newenvironment{problem}[2][Problem]{\begin{trivlist}
\item[\hskip \labelsep {\bfseries #1}\hskip \labelsep {\bfseries #2.}]}{\end{trivlist}}

\begin{document}

\title{Problem Set 3, Answers}
\author{Kevin Lacker}
\maketitle

\begin{problem}{1}
  This is a four-part problem.
  \begin{enumerate}
    \item Let $L$ be the set of leaves of the minimum-height protocol
      tree for $f$ that
      give output 1. For each $l \in L$, let $M_l$ be the matrix which
      is 1 on the input pairs whose path through the protocol tree
      ends at the lead $l$. By the definition of a protocol tree,
      $M_l$ can also be defined by a set $X$ and $Y$ with:
      \begin{equation}
        M_l[i,j] = 1 \iff i \in X, j \in Y
      \end{equation}
      Thus, there are two possible rows in
      $M_l$, either all zeros or 1 precisely on indices in $Y$. Thus,
      the rank of $M_l$ is 1, regardless of what field the rank is
      taken over.

      Note that:
      \begin{equation}
        M_f = \sum_{l \in L} M_l
      \end{equation}

      Since rank is subadditive, we can conclude:

      \begin{equation}
        rank(M_f) \leq \sum_{l \in L} rank(M_l) = |L| \leq 2^{D(f)}
      \end{equation}

      The last inequality holds since $L$ is a subset of leaves of a
      tree of height $D(f)$. We can then conclude

      \begin{equation}
        log_2(rank(M_f)) \leq D(f)
      \end{equation}
    \item There are $2^n$ rows of $M_f$. Consider them as vectors over the field
      $\mathbb{F}_2$, and say the rows span a
      subspace of dimension $d$, so $rank(M_f) = d$. Since the field has size 2, this
      subspace has exactly $2^d$ elements in
      it. Since the rows of $M_f$ are $2^n$ distinct vectors in this
      subspace of size $2^d$, we have $2^n \leq 2^d$. So
      $log_2(n) \leq log_s(d)$ and by the conclusion of the previous
      problem, $\log_2(n) \leq D(f)$.
    \item TODO
    \item TODO
  \end{enumerate}
\end{problem}

\begin{problem}{2}
  Our randomized protocol for GT will use the randomized protocol for
  EQ, to do a binary search to find the first bit in which the two bit
  strings disagree. The algorithm works like:
  \begin{itemize}
  \item Find the middle of our bit string.
  \item Use the EQ protocol to determine whether the bits to the
    left of the midpoint are the same in the two strings.
  \item If they are the same, recurse on the right half of the
    string. If they are not the same, recurse on the left half of
    the string.
  \item Once we have found the first bit that disagrees, in
    constant communication we can figure out which of the
    strings is greater. 
  \end{itemize}
  We have $O(log n)$ steps, and each step introduces a constant
  error. We need to reduce this error to $O(1 / log(n))$ so that it is
  constant over the whole algorithm, so we need to
  repeat the protocol at each steg $O(log(log(n)))$ times. This gives a
  total communication complexity of $R^{pub}(GT) \leq O(log(n)
  log(log(n)))$.

  Newman's theorem states that this can be converted into a private
  coin protocol with an additive log penalty, but this complexity is
  larger than the log penalty already, so the same asymptotic bound holds for
  $R(GT)$ as for $R^{pub}(GT)$.
\end{problem}

\end{document}
