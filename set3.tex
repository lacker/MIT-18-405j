\documentclass{article}
\usepackage{amsmath,amsthm,amssymb}

\newenvironment{problem}[2][Problem]{\begin{trivlist}
\item[\hskip \labelsep {\bfseries #1}\hskip \labelsep {\bfseries #2.}]}{\end{trivlist}}

\begin{document}

\title{Problem Set 3, Answers}
\author{Kevin Lacker}
\maketitle

\begin{problem}{1}
  This is a four-part problem.
  \begin{enumerate}
    \item Let $L$ be the set of leaves of the minimum-height protocol
      tree for $f$ that
      give output 1. For each $l \in L$, let $M_l$ be the matrix which
      is 1 on the input pairs whose path through the protocol tree
      ends at the lead $l$. By the definition of a protocol tree,
      $M_l$ can also be defined by a set $X$ and $Y$ with:
      \begin{equation}
        M_l[i,j] = 1 \iff i \in X, j \in Y
      \end{equation}
      Thus, there are two possible rows in
      $M_l$, either all zeros or 1 precisely on indices in $Y$. Thus,
      the rank of $M_l$ is 1, regardless of what field the rank is
      taken over.

      Note that:
      \begin{equation}
        M_f = \sum_{l \in L} M_l
      \end{equation}

      Since rank is subadditive, we can conclude:

      \begin{equation}
        rank(M_f) \leq \sum_{l \in L} rank(M_l) = |L| \leq 2^{D(f)}
      \end{equation}

      The last inequality holds since $L$ is a subset of leaves of a
      tree of height $D(f)$. We can then conclude

      \begin{equation}
        log_2(rank(M_f)) \leq D(f)
      \end{equation}
    \item There are $2^n$ rows of $M_f$. Consider them as vectors over the field
      $\mathbb{F}_2$, and say the rows span a
      subspace of dimension $d$, so $rank(M_f) = d$. Since the field has size 2, this
      subspace has exactly $2^d$ elements in
      it. Since the rows of $M_f$ are $2^n$ distinct vectors in this
      subspace of size $2^d$, we have $2^n \leq 2^d$. So
      $log_2(n) \leq log_s(d)$ and by the conclusion of the previous
      problem, $\log_2(n) \leq D(f)$.
    \item TODO
    \item TODO
  \end{enumerate}
\end{problem}

\begin{problem}{2}
  Our randomized protocol for GT will use the randomized protocol for
  EQ, to do a binary search to find the first bit in which the two bit
  strings disagree. The algorithm works like:
  \begin{itemize}
  \item Find the middle of our bit string.
  \item Use the EQ protocol to determine whether the bits to the
    left of the midpoint are the same in the two strings.
  \item If they are the same, recurse on the right half of the
    string. If they are not the same, recurse on the left half of
    the string.
  \item Once we have found the first bit that disagrees, in
    constant communication we can figure out which of the
    strings is greater. 
  \end{itemize}
  We have $O(log n)$ steps, and each step introduces a constant
  error. We need to reduce this error to $O(1 / log(n))$ so that it is
  constant over the whole algorithm, so we need to
  repeat the protocol at each steg $O(log(log(n)))$ times. This gives a
  total communication complexity of $R^{pub}(GT) \leq O(log(n)
  log(log(n)))$.

  Newman's theorem states that this can be converted into a private
  coin protocol with an additive log penalty, but this complexity is
  larger than the log penalty already, so the same asymptotic bound holds for
  $R(GT)$ as for $R^{pub}(GT)$.
\end{problem}

\begin{problem}{3}
  Assuming $f$ has a protocol tree $T$ with $l$ leaves, our task is to
  construct another protocol tree for $f$ that has depth $O(l)$.

  First, let's find a subtree of $T$ that contains approximately half
  of its nodes. Start at the root and at each step recurse toward the
  larger subtree. The size of the subtree can drop by a factor of at
  most $1/2$ in each iteration. It starts at $l$ and ends at
  1. Therefore, it must pass through the range of $l/3$ to $2l/3$ at
  some point. We pick the subtree in that range and call it $U$.

  So $T$ can be seen as the combination of two trees, $U$ and
  $T - U$. Each of these trees is at most $2/3$ the size of $T$. Each
  node in $T$ corresponds to a rectangle, so there are two sets $X$
  and $Y$ such that inputs $(x, y)$ are determined by $T$ iff
  $x \in X$ and $y \in Y$.

  We can also use recursion to find two balanced trees, $B_U$ and
  $B_{T-U}$, which are protocol trees for $U$ and $T-U$ respectively,
  with depth is less than $c \cdot log(2l/3)$ for some constant $c$.

  Now, we can construct a new protocol tree for $f$.

  \begin{itemize}
    \item Determine whether the
      inputs are in the $(X, Y)$ rectangle. This just takes two bits
      of communication, one from each player whether their input is in
      the rectangle.
    \item If they are in the rectangle, use $B_U$.
    \item If they are not in the rectangle, use $B_{T-U}$.
  \end{itemize}

  The total depth of this tree is $2 + c \cdot log(2l/3)$, which for large
  enough $c$ is less than $c \cdot log(l)$. So there is a protocol tree for
  $f$ with $O(log(l))$ depth.
\end{problem}

\begin{problem}{4}
  Rather than solve this problem once using $O(log^2(n))$ and again in
  $O(log(n))$, we'll just do it the harder way, which implies the
  easier way.
  
  The first step is to use a log amount of communication to reduce the
  problem to one where the sets are either the same size or differ in
  size by 1 element. Alice and Bob can
  communicate the size of their sets with log communication. Without
  loss of generality, assume that Bob has more elements. Bob then
  calculates the interval $b_{low}$ to $b_{high}$ that contains his
  median elements that would be the same amount of elements, plus or
  minus 1, as Alice has. So the number of elements Bob has below
  $b_{low}$ and the number of elements above $b_{high}$ are equal. Bob
  then communicates these bounds to Alice.

  If Alice's elements are either all above $b_{low}$ or all above
  $b_{high}$, she communicates this fact to Bob. Bob is then able to
  calculate the median all by himself, because Alice's elements are
  all known to lie on one particular side of the median.

  Otherwise, we know the median lies within the $b_{low}$ and
  $b_{high}$ interval. Bob can then discard his elements outside that
  interval without affecting the median, and the algorithm can
  continue.

  Now we need a recursion phase. Alice and Bob have $O(n)$
  elements (same or off by 1), and they know the median lies in a
  particular range of size $O(n)$. Alice and Bob pick the midpoint of
  the range and tell each other whether most of their items are above
  or below of the midpoint, or whether they are split evenly. There
  are a few possible outcomes.

  \begin{enumerate}
  \item If the majority of Alice's and Bob's items are on the same
    side of the midpoint, we know the median lies on that side as
    well. Recurse on the smaller range. This also works if one player
    has their items evenly split, and the other player has a majority.
  \item If the majority of Alice's and Bob's items are on opposite
    sides of the midpoint, they can both discard half of their items
    without changing the midpoint. (If the amounts are off by 1,
    round so that each player discards the same amount.) Recurse on
    the same range, but with fewer elements.
    \item If both Alice and Bob have their items evenly split on each
      side of the midpoint, we know the median is the average of the
      highest item below the midpoint and the lowest item above it. So
      Alice and Bob can each report their closest items to the
      midpoint, both above and below, and this $O(log(n))$ communication is enough
      to calculate the median, without recursing.
  \end{enumerate}

  Each phase of the recursion takes constant communication and cuts
  either the size of the sets or the size of the range by a constant
  fraction, so the total communication is $O(log(n))$.
\end{problem}

\begin{problem}{5}
  \begin{enumerate}
  \item Here is a recursive algorithm for finding
    $|C \cap I|$. Note that it must be either 0 or 1 because only a
    zero-vertex or single-vertex subgraph can be both a clique and an independent
    set.

    \begin{itemize}
      \item Alice finds the vertex $v_A$ in $C$ with the smallest degree. If
        the degree is no more than $n/2$, communicate this vertex to
        Bob. If Bob has this vertex, we are done. If Bob does not have
        this vertex, we have at least communicated that the answer
        must be adjacent to $v_A$. Discarding the vertices that are not
        adjacent to $v_A$, we can recurse on the rest.
      \item Bob then finds the vertex $v_B$ in $I$ with the largest
        degree. If the degree is no less than $n/2$, communicate
        this vertex to Alice. If Alice has this vertex, we are
        done. If Alice does not have this vertex, we have at least
        communicated that the answer cannot be adjacent to
        $v_B$. Discarding the vertices that are adjacent to $v_B$,
          we can recurse on the rest.
      \item If neither Alice nor Bob has a vertex satisfying these
        conditions, we know none of their vertices can overlap, since
        the possible range of their degrees does not overlap. So we
        can return 0.
    \end{itemize}

    The recursion takes at most $O(log(n))$ steps, and each step has
    communication of size $O(log(n))$, so the total communication for
    this algorithm is $O(log^2(n))$.

    \item TODO: part 2
  \end{enumerate}

\end{problem}
  
\end{document}
