\documentclass{article}
\usepackage{amsmath,amsthm,amssymb}

\newenvironment{problem}[2][Problem]{\begin{trivlist}
\item[\hskip \labelsep {\bfseries #1}\hskip \labelsep {\bfseries #2.}]}{\end{trivlist}}

\begin{document}

\title{Problem Set 2, Answers}
\author{Kevin Lacker}
\maketitle

\begin{problem}{1}
  The idea is to use MAJ circuits to calculate PARITY, and thus use
  our lower bound on PARITY circuits to provide a lower bound on MAJ
  circuits.
  
  First, we can create a circuit for the function that counts whether
  precisely half of the bits of the input are 1. I.e., $HALF(x) = 1$ iff
  $|x| = n / 2$, with two MAJ circuits:

  \begin{equation}
    HALF(x) = MAJ(x) \wedge MAJ(\neg x)
  \end{equation}

  Here $\neg x$ represents the bitwise negation of $x$. Our circuit for
  for HALF uses two MAJ subcircuits and one more depth.

  Now that we have HALF, we can construct a circuit to count whether
  precisely $k$ bits of the input are 1, $EXACT_k$, by padding the input with
  ones or zeros and using a single HALF circuit on at most twice the
  input size.

  We can then take one $EXACT_K$ for each odd number less than or
  equal to $n$, and take their conjunction to create a PARITY circuit
  of depth $d+2$. This PARITY circuit of depth $d+2$ is built of
  $O(n)$ depth-$d$ MAJ circuits, plus $O(n)$ extra gates, where each
  MAJ circuit has at most $2n$ inputs.

  Let $H_{MAJ}(n, d)$ denote the minimum size of a circuit of depth
  $d$ calculating MAJ on a size-$n$ input, and $H_{PARITY}$ the same
  for PARITY. This construction demonstrates that

  \begin{equation}
    O(n) \cdot H_{MAJ}(n, d) \geq H_{PARITY}(n/2, d + 2)
  \end{equation}

  But we know that

  \begin{equation}
    H_{PARITY}(n, d) \geq exp(\Omega(n^{2^{-d}}))
  \end{equation}

  Substituting in and simplifying we get

  \begin{equation}
    H_{MAJ}(n, d) \geq exp(\Omega(n^{2^{-d-O(1)}}))
  \end{equation}
  
\end{problem}

\begin{problem}{2}
  Let's say we have an estimate $v$ of $\Sigma_xf(x)$ and we want to
  improve that estimate. We define:
  \begin{equation}
    g(x) =
    \begin{cases}
      f(0) - v & \text{for } x = 0 \\
      f(x) & \text{otherwise}
    \end{cases}
  \end{equation}

  We can then use our estimation oracle to get an estimation for
  $\Sigma_xg(x)$ = $\Sigma_xf(x) - E$. Basically, we are estimating
  the error in our original estimate. When we add this estimate to
  $E$, we are improving our estimate to $\Sigma_xf(x)$. If our
  estimation oracle returns a value within a factor of
  $1 \pm \epsilon$ of the correct value, then the size of our error
  shrinks by a factor of $\epsilon$ every iteration.

  Recursing until our error is below 1 thus takes time that is linear
  in the size of the output value, so we can use this to get an exact
  answer to $\Sigma_xf(x)$ in polynomial time, assuming the answer has
  polynomial length.

  We can use this to exactly solve problems in $\sharp SAT$. Let $x$
  represent a possible solution to a $\sharp SAT$ problem, and $f(x) = 1$
  when $x$ is a solution, $0$ otherwise. Since $f(x)$ is a constant,
  the sum has polynomial length, and our algorithm takes polynomial
  time.

  $\sharp SAT$ is $\sharp P$-complete so this solves any $\sharp P$ problem in
  polynomial time.
\end{problem}

\begin{problem}{3}
  Say we have a polynomial that agrees with $OR$ over $\{0,1\}^n$ with
  degree less than n, $P_{OR}(x_0, x_1, ... x_n) = OR(x_0, x_1, ... x_n)$.
  Then we can substitute variables to get another polynomial with
  degree less than $n$ that represents $AND$:

  \begin{equation}
    P_{AND}(x_0, x_1, ..., x_n) = 1 - P_{OR}(1 - x_0, 1 - x_1, ..., 1 - x_n)
  \end{equation}

  Now consider any string of $n$ bits, $y = y_0y_1...y_n$. We can
  construct a polynomial of degree less than $n$ that is $1$ on this
  input and $0$ on all the other inputs in $\{0, 1\}^n$:

  \begin{equation}
    P_y(x_0, x_1, ..., x_n) = P_{AND}(x_0 - y_0, x_1 - y_1, ..., x_n - y_n)
  \end{equation}

  Now these functions act as a basis if you think of this as a vector
  space. So any function $f(x)$ from
  $\{0, 1\}^n \rightarrow \mathbb{Z}_q^n$ can be represented as a
  polynomial of degree less than $n$:

  \begin{equation}
    f(x) = \Sigma_{y \in \{0, 1\}^n}P_y(x)f(y)
  \end{equation}

  There are $q^{2^n}$ different such functions, as defined by their
  values on $\{0, 1\}^n$. But there are only $q^{2^d}$ different
  polynomials of degree $d$, because such polynomials can have only
  $2^d$ different terms. This means $d$ cannot be less than $n$, which
  gives us a contradiction.
\end{problem}

\begin{problem}{4}
  Let's first do the direction that every language in $PH$ can be
  computed by an exponential-size, constant-depth uniform circuit.

  We will proceed by induction. For the base case, we want a
  constant-depth circuit for any language in $P$. First, we want a
  layer of $2^n$ $AND$ gates, one gate for each possible input. Each
  gate is activated only on a single input. Our second and final layer is
  a single $OR$ gate. When we calculate the $EDGE$
  function, we just calculate whether the $i$th input string is in our
  language or not, and if so, we add an edge to the $AND$ gate that is
  activated precisely by that input string. This depth-$2$ circuit
  computes our language in $P$.

  Note that these circuit families can be negated by switching every $AND$
  gate to an $OR$ gate, and negating every input. Thus, if we can
  represent $\Sigma_k^P$ languages by these circuits, we can represent $\Pi_k^P$
  languages by these circuits, and vice versa.

  Now, consider a language $L \in \Sigma_k^P$. It can be written as

  \begin{equation}
    x \in L \iff \exists u : ux \in L'
  \end{equation}

  where $L'$ is in $\Pi_{k-1}^P$. We recursively find a circuit family
  that calculates $L'$, and make $2^n$ copies of that circuit, one for
  every possible $x$. Each of these copies can be calculated in
  polynomial-time, so the resulting circuit is still DC uniform. We
  combine all of those with a single $OR$ gate to create a circuit
  that calculates $L$. Induction then tells us that these circuits can
  compute every language in $PH$.

  For the other direction, it's simpler to see that an alternating
  Turing machine can evaluate a DC uniform circuit with a constant
  number of alternations. When we hit an $AND$ node, use the mode that
  accepts when every branch accepts, and have one branch go to each
  child of the $AND$ node. When we hit an $OR$ node, use the mode that
  accepts when any branch accepts, and have one branch go to each
  child of the $OR$ node. Since each local graph area can be computed
  in polynomial time, and the graph has constant depth, the whole
  algorithm will run in polynomial time. Since depth is constant, we
  only need a constant number of alternations. So a DC uniform circuit
  can be evaluated by an alternating Turing machine in a constant
  number of alternations, and $PH$ is powerful enough to compute these
  circuits. Thus the models are equivalent.
\end{problem}

\begin{problem}{5}
  This is a three-part problem.
  \begin{enumerate}
    \item SUCCINCT-3SAT is $NEXP$-complete, so we just have to show
      that SUCCINCT-3SAT is in $E^{NP}$. This is straightforward - it
      takes exponential time to expand the SAT instance from its
      compressed form, and then we can solve it with a single call to
      the $NP$ oracle.
    \item $NE \subset NEXP$, so
      $NEXP \nsubseteq P/poly \implies NE \nsubseteq P/poly$.
    \item We can construct the truth table for SUCCINCT-3SAT. There
      are $2^n$ different compressed 3SAT problems possible on size
      n. So, we can nondeterministically guess an answer to each
      problem, and put $0$ in the truth table if it is not a
      correct answer, $1$ if it is the correct answer. This takes
      poly($2^n$) time for each problem, so poly($2^n$) time overall.

      However, this is not enough. We still need to figure out which
      of the branches should be the accepting branch. To do this, our
      advice string can tell us how many $1$'s should be in the truth
      table. Note that our algorithm has one-sided error - it only
      writes a $0$ where there should be a $1$, not vice versa. So if
      we only accept when we have the correct number of $1$'s in the
      truth table, that is sufficient to limit to a single accepting
      branch.

      Since SUCCINCT-3SAT is NEXP-complete, by our assumption there is
      no polynomial-size circuit family that can calculate it, and
      these truth tables satisfy the desired circuit complexity condition.
  \end{enumerate}
\end{problem}

\end{document}
