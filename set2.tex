\documentclass{article}
\usepackage{amsmath,amsthm,amssymb}

\newenvironment{problem}[2][Problem]{\begin{trivlist}
\item[\hskip \labelsep {\bfseries #1}\hskip \labelsep {\bfseries #2.}]}{\end{trivlist}}

\begin{document}

\title{Problem Set 2, Answers}
\author{Kevin Lacker}
\maketitle

\begin{problem}{1}
  The idea is to use MAJ circuits to calculate PARITY, and thus use
  our lower bound on PARITY circuits to provide a lower bound on MAJ
  circuits.
  
  First, we can create a circuit for the function that counts whether
  precisely half of the bits of the input are 1. I.e., $HALF(x) = 1$ iff
  $|x| = n / 2$, with two MAJ circuits:

  \begin{equation}
    HALF(x) = MAJ(x) \wedge MAJ(\neg x)
  \end{equation}

  Here $\neg x$ represents the bitwise negation of $x$. Our circuit for
  for HALF uses two MAJ subcircuits and one more depth.

  Now that we have HALF, we can construct a circuit to count whether
  precisely $k$ bits of the input are 1, $EXACT_k$, by padding the input with
  ones or zeros and using a single HALF circuit on at most twice the
  input size.

  We can then take one $EXACT_K$ for each odd number less than or
  equal to $n$, and take their conjunction to create a PARITY circuit
  of depth $d+2$. This PARITY circuit of depth $d+2$ is built of
  $O(n)$ depth-$d$ MAJ circuits, plus $O(n)$ extra gates, where each
  MAJ circuit has at most $2n$ inputs.

  Let $H_{MAJ}(n, d)$ denote the minimum size of a circuit of depth
  $d$ calculating MAJ on a size-$n$ input, and $H_{PARITY}$ the same
  for PARITY. This construction demonstrates that

  \begin{equation}
    O(n) \cdot H_{MAJ}(n, d) \geq H_{PARITY}(n/2, d + 2)
  \end{equation}

  But we know that

  \begin{equation}
    H_{PARITY}(n, d) \geq exp(\Omega(n^{2^{-d}}))
  \end{equation}

  Substituting in and simplifying we get

  \begin{equation}
    H_{MAJ}(n, d) \geq exp(\Omega(n^{2^{-d-O(1)}}))
  \end{equation}
  
\end{problem}

\begin{problem}{2}
  TODO
\end{problem}

\end{document}
