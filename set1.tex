\documentclass{article}
\usepackage{amsmath,amsthm,amssymb}

\newenvironment{problem}[2][Problem]{\begin{trivlist}
\item[\hskip \labelsep {\bfseries #1}\hskip \labelsep {\bfseries #2.}]}{\end{trivlist}}

\begin{document}

\title{Problem Set 1, Answers}
\author{Kevin Lacker}
\maketitle

\begin{problem}{1}

  A language $L$ is in $\Sigma_2^{\bf{P}}$ iff there is a polynomial time TM
  $M$ such that:

  \begin{equation}
    x \in L \iff \exists u_1 \forall u_2 M(x, u_1, u_2)
  \end{equation}

  where the $u_i$ are polynomial size and we treat $M$ as returning
  true or false. This can also be written as:

  \begin{equation}
    x \in L \iff \exists u_1 \neg ( \exists u_2 ( \neg M(x, u_1, u_2) )
  \end{equation}

  The answer to $\exists u_2 ( \neg M(x, u_1, u_2) )$ can be found
  in a single call to an $\bf{NP}$ oracle, so $\Sigma_2^{\bf{P}} \in
  \bf{NP}^{\bf{NP}}$ .

  The other direction is similar. TODO: prove the other direction as well
  
\end{problem}


\end{document}
